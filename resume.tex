\documentclass[letterpaper,12pt]{article}
\pagestyle{empty}

\usepackage[usenames,dvipsnames]{color}
\usepackage[colorlinks,breaklinks,urlcolor=Green]{hyperref}

% For global font and paragraph settings
\usepackage{lmodern}
\renewcommand*\familydefault{\sfdefault}
\renewcommand*\sfdefault{lmss}
\setlength\parindent{0in}

% For margins
\usepackage[margin=0.75in]{geometry}

% For CV section headings
\usepackage{sectsty}
\sectionfont{\mdseries\scshape\raggedright\large}
\subsectionfont{\mdseries\itshape\raggedright\normalsize}

\begin{document}

{\Huge\textsc{Hana Lee}}\\
\rule[1mm]{\textwidth}{0.5pt}

\footnotesize\begin{center}
    \href{hanalee07@gmail.com}{hanalee07@gmail.com}
    \textbullet\enskip\href{http://hanalee.info}{hanalee.info}
    \textbullet\enskip\textbf{github} \href{https://github.com/hnlee}{hnlee}
    \textbullet\enskip\textbf{linkedin} \href{https://www.linkedin.com/in/hanalee07}{hanalee07}
\end{center}

\section*{Summary}
\begin{itemize}
    \item Scientist interested in exploring and analyzing big data to solve meaningful problems.
    \item Former career as a genetics and genomics researcher, with expertise in analysis of next-generation sequencing data sets and scientific computing for the life sciences.
\end{itemize}

\section*{Roles and experiences}

 

\subsection*{Data analysis and visualization}
\begin{itemize}
    \item Built data cleaning and processing pipelines in Python for terabyte-scale data sets
    \item Used R for exploratory data analysis and statistical testing, including nonparametric methods, generalized linear models, and multiple testing correction
    \item Wrote Python script to semiautomate colony counting on images
    \item Rewrote R code developed by collaborator in Python with data storage in SQLite for faster implementation
 \end{itemize}
 
\subsection*{Teaching and communication}
\begin{itemize} 
    \item Introduced undergraduate and doctoral students to programming through Python and R
    \item Co-taught a workshop for scientists on analyzing next-generation sequencing data in R
    \item Expert at communicating complex information to nonspecialist audiences through data visualization and oral presentations
\end{itemize}

\subsection*{Machine learning projects}
\begin{itemize}
    \item \textbf{San Francisco crime classification}
        \begin{itemize}
            \item Developed model averaging predictions from logistic regression and random forests using Python as project for Chicago Python User Group (ChiPy) Fall 2015 Mentorship Program
            \item Submitted to Kaggle competition and placed in top 25\% of the leaderboards
            \item Presented work at the ChiPy January 2016 meeting (see \href{http://hnlee.github.io/sfcrimes/#/}{slides})
        \end{itemize}
    \item \textbf{Gas mileage prediction:} Developed generalized linear model trained on fuel economy data set using R, and created webapp using \texttt{shiny} to deliver mileage predictions and display plots in response to user input (see \href{https://hnlee.shinyapps.io/shinyapp/}{webapp})
    \item \textbf{Bicep curl classification:} Developed random forests model trained on accelerometer data set using R, and wrote report in \texttt{knitr} (see \href{http://hnlee.github.io/pmlCourseProject/}{report}) 
\end{itemize}
 
\subsection*{Web development}
\begin{itemize}
    \item Maintained a donations website written in \texttt{web2py} for alumni organization
    \item Created backend in \texttt{django} with SQLite database for a website to manage event invitations
    \item Built web interface in \texttt{django} with MySQL database to crowdsource and display media metadata
\end{itemize}

\section*{Education}
\begin{itemize}
    \item 2012 \textbf{Ph.D.} Molecular Biology, University of California, Berkeley
    \item 2007 \textbf{A.B.} Biochemical Sciences, Harvard University
\end{itemize}
\end{document}
