\documentclass[letterpaper,12pt]{article}
\pagestyle{empty}

% For publications list
\usepackage[usenames,dvipsnames]{color}
\usepackage{natbib}
\usepackage[colorlinks,breaklinks,urlcolor=Green]{hyperref}
\renewcommand\refname{Publications}

% For global font and paragraph settings
\usepackage{lmodern}
\renewcommand*\familydefault{\sfdefault}
\renewcommand*\sfdefault{lmss}
\setlength\parindent{0in}

% For margins
\usepackage[margin=1in]{geometry}

% For CV section headings
\usepackage{sectsty}
\allsectionsfont{\mdseries\scshape\raggedright\large}

% For header
\newcommand{\header}[2]{
    \parbox[b][3em][c]{0.3\textwidth}{
        \Huge\textsc{#1}
    }
    \rule[0pt]{\textwidth}{0.5pt}
    \parbox[c]{\textwidth}{
        \centering{#2}
    }
} 

\begin{document}

\header{Hana Lee}{
    \textbf{email}\href{mailto:hanalee07@gmail.com}{hanalee07@gmail.com}
    \bullet
    \textbf{website}\href{http://www.hanalee.info}{www.hanalee.info}
    \bullet
    \textbf{github}\href{https://github.com/hnlee}{hnlee}
    \bullet
    \textbf{linkedin}\href{https://www.linkedin.com/in/hanalee07}{hanalee07}
}

\section*{Summary}
\begin{itemize}
    \item Scientist interested in exploring and analyzing big data, currently seeking position as a data scientist.
    \item Former career as a genetics and genomics researcher, with expertise in analysis of next-generation sequencing data sets and scientific computing for the life sciences.
\end{itemize}

\section*{Skills}
\begin{itemize}
    \item Scientific computing and visualization: \textbf{R} and \textbf{Python} 
        \begin{itemize}
            \item R packages: \texttt{caret}, \texttt{ggplot2}, \texttt{shiny}, \texttt{knitr}
            \item Python packages: \texttt{numpy}, \texttt{pandas}, \texttt{scikit-learn}, \texttt{seaborn}, \texttt{jupyter}
        \end{itemize}
    \item Relational databases: \textbf{SQLite} and \textbf{MySQL}
    \item Web framework: \textbf{Django}
\end{itemize}

\section*{Education}
\begin{itemize}
    \item 2012 \textbf{Ph.D.} Molecular Biology, University of California, Berkeley
    \item 2007 \textbf{A.B.} Biochemical Sciences, Harvard University
\end{itemize}

\section*{Roles and experiences}
\begin{itemize}
    \item \textbf{Data analysis and visualization:} In my graduate student researcher and postdoctoral fellow positions, I routinely built data cleaning and processing pipelines in Python for large genomic data sets and used R for exploratory data analysis and statistical testing. I have experience with nonparametric tests and generalized linear models, as well as using SQL databases for data management.
    \item \textbf{Teaching:} In the process of supervising technicians and student researchers, I have frequently taught them basic Python and R as part of their training. I have also co-taught a one-day workshop on using R to analyze next-generation sequencing data.
    \item \textbf{Communication:} As an interdisciplinary researcher, I am experienced in communicating complex information to nonspecialist audiences, in the form of data visualization and oral presentations.
    \item \textbf{Self-directed learning:} I have worked on several data science projects taking MOOC on machine learning and participating in the Chicago Python User Group's mentorship program, including: developing a classification model to predict exercise movements from accelerometer data, building a \texttt{shiny} webapp to deliver predictions on gas mileage based on a linear regression model, and working on various challenges, such as the Yelp Dataset Challenge and Kaggle.
\end{itemize}

\end{document}
